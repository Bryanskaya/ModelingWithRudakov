На Листинге \ref{code} представлены основные методы.

\begin{lstlisting}[label=code, caption = Основные методы]
namespace calculations {
	class MathCalc {
		public static double Criteria(int[] arr, int elMin, int elMax) {
			double y = 0, p = 1.0 / (elMax - elMin);
			
			for (int i = elMin; i < elMax; i++)
				y += Math.Pow(arr.Count(x => x == i), 2) / p;
			y = y / (double)arr.Length - arr.Length;
			return ChiSquared.CDF(elMax - elMin - 1, y);
		}
	}

	class CongruentMethod {
		private int a { get; }
		private int b { get; }
		private int N { get; }
		private int r;
		
		public CongruentMethod(int seed) {	
			a = 1103515245;
			b = 12345;
			r = seed;
		}
		
		public int next(int start, int end){
			r = a * r + b;
			return (int)((uint)r >> 16) % (end - start) + start;
		}
	}
}

namespace lab03 {
	public partial class Form1 : Form {
		...
		private void _fillAlg(int start, int end, string filename, string field, string fieldK){
			int[] arr = new int[1000];
			CongruentMethod alg = new CongruentMethod(rnd.Next());
			
			for (int i = 0; i < arr.Length; i++)
				arr[i] = alg.next(start, end);
			...
			fillCriteria(fieldK, MathCalc.Criteria(arr, start, end));
		}
		
		private void _useTable(int start, int end, string filename, string field, string fieldK){
			int[] data = new int[1000], arr = new int[1000];
			int num = rnd.Next(0, 1000), j = num;
			
			_getArrFromFile(filename, ref data);
			
			for (int i = 0; i < arr.Length; i++) {
				arr[i] = data[j];
				j = (j + 1) % data.Length;
			}
			fillCriteria(fieldK, MathCalc.Criteria(arr, start, end));
		}
	
		private void processUser(){
			int[] arr = new int[10];
			int flag, start, end;
			string fieldK;
			
			try{
				_getArrFromUser(out flag, ref arr);
			}
			catch (Exception e){
				...
			}
			
			switch (flag){
				case 1:
					start = 0;
					end = 10;
					fieldK = "criteriaO1";
					break;
				case 2:
					start = 10;
					end = 100;
					fieldK = "criteriaO2";
					break;
				default:
					start = 100;
					end = 1000;
					fieldK = "criteriaO3";
					break;
			}
			fillCriteria(fieldK, MathCalc.Criteria(arr, start, end));
		}
	}
}

\end{lstlisting}
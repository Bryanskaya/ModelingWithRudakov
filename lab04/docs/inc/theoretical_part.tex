\subsection{Равномерное распределение}
Плотность выражается формулой:
\begin{equation}
	f_X(x) = 
	\left\{\begin{array}{l}
		\frac{1}{b - a}, x \in [a, b], \\
		0, x \notin [a, b].
	\end{array}\right.
\end{equation}

Функция распределения имеет вид:
\begin{equation}
	F_X(x) = 
	\left\{\begin{array}{l}
		0, x < a, \\
		\frac{x - a}{b - a}, a \leq x < b, \\
		1, x \geq b.
	\end{array}\right.
\end{equation}

\subsection{Нормальное распределение}
Плотность выражается формулой:
\begin{equation}
	f_X(x) = \frac{1}{\sigma \sqrt{2 \pi}} e^{-\dfrac{(x - \mu)^2}{2 \sigma^2}}
\end{equation}

Функция распределения имеет вид:
\begin{equation}
	F_X(x) = \frac{1}{\sigma \sqrt{2 \pi}} \int\limits_{-\infty}^x e^{-\dfrac{(x - \mu)^2}{2 \sigma^2}} \,dx
\end{equation}

Стандартным нормальным распределением называется нормальное распределение с математическим ожиданием $\mu = 0$ и стандартным отклонением $\sigma = 1$.

\subsection{Пошаговый принцип ($\Delta t$)}
Этот принцип заключается в последовательном анализе состояний всех блоков системы в момент $t + \Delta t$. При этом новое состояние блоков определяется в соответствии с их алгоритмическим описанием. 

Недостаток: значительные временные затраты на реализацию моделирования системы. А также при недостаточно малом $\Delta t$ отдельные события в системе могут быть пропущены, что может повлиять на адекватность результатов.

\subsection{Событийный принцип}
Состояние отдельных устройств изменяются в дискретные моменты времени, совпадающие с моментами времени поступления сообщений в систему, временем окончания обработки задачи и т.д.

При использовании событийного принципа состояние всех блоков системы анализируется лишь в момент проявления какого-либо события. Моменты наступления следующего события определяются минимальным значением из списка событий.

